\part{新约}


\chapter*{新约全书导论}
\addcontentsline{toc}{chapter}{新约全书导论}
“新约全书”是\UL{耶稣}死后,由其宗徒弟子,在天主圣神的默感与引导之下,所写成的经典汇集。此汇集由第二世纪起即称为《新约书》,或简称《新约》。称之为“约”,因为其中所讲论的,是天主与人类所立的盟约;称之为“新”,以别于“旧约”。“旧约”是天主与\UL{以}民在\UL{西乃}山上所立的圣约,而“新约”是\UL{基督}以自己的圣血与圣死,在天主与人间,所建立的救恩圣约(参阅\uwave{玛}26:28;\uwave{谷}14:24等处)。

“新约全书”,按圣教会古老的传授,共计二十七卷;

《历史书》五卷:《玛窦福音》、《马尔谷福音》、《路加福音》、《若望福音》和《宗徒大事录》。

《训诲书》二十一卷:圣\UL[保禄]快十四封:《罗马书》、《格林多》前后二书、《迦拉达书》、《厄弗所书》、《斐理伯书》、《哥罗森书》、《得撒洛尼》前后二书、《弟茂德》前后二书、《弟铎书》、《费肋孟书》和《希伯来书》;公函七封:《雅各伯书》、《伯多禄》前后二书、《若望》一、二、三书并《犹达书》。

《先知书》一卷:《若望默示录》。

《新约全书》,除《玛窦福音》的原文为\UL[阿刺美]文外,都是用\UL[希腊]文写成的。这些似乎有些奇怪,因为按当时\UL[耶稣]在世时,和宗徒最初讲道时所用的语言,本来都是\UL[阿刺美]语,并且全部《新约》作者,除圣\UL{路加}外,又都是\UL[犹太]人;那么为什么不用本国文字编写呢?其理由是因为只有《玛窦福音》是写给\UL[巴力斯坦]的\UL[犹太]人,而其余的书都是写给说\UL[希腊]话的基督徒,其中很少有通晓\UL[阿刺美]语的;更何况《新约》又是向天下万民所公布的;因此以当时\UL[罗马]帝国内所通行的\UL[希腊]语编写,是很自然的事。

《新约全书》(或《新经》),就宗教方面来说,远远超过《旧约全书》(或《古经》),因为天主在旧约时代只是“多次并以多种方式,藉着先知对祖先说过话”;然而在新约时代却是“藉着了对我们说了话”(\uwave{希}1:1)。如此,旧约的启示在新约内才得以圆满;旧约的预许在新约内才得以实现。所以吾人除非认识《新约》,决不能完全明了《旧约》;为此,可说《新约全书》实是世界上最重要和最宝贵的作品。


\chapter*{福音总论}
\addcontentsline{toc}{chapter}{福音总论}
“福音”一词,按其字音,原指“喜讯”;但按《新约》作者彩此词的意义来说,乃是指天主子\UL{耶稣}隆重为人,从天上给人类带来的启示,和在他完成救赎工程以后,诸宗徒向万民所宣布的得救喜讯。

这喜讯的传报,最初只靠口头的宣讲,稍后才有不少人士把\UL{耶稣}的生平与宣讲笔之于书,因而产生了“福音”的著作。按\uwave{路}1:1的记载,这样的著作在当时已为数不少,可是圣教会自初只承认《玛窦》、《马尔谷》、《路加》、《若望》这四部《福音》为受默感而写的经典,并著录在正经书目内,其他名为“福音”的著作,概著录为伪经。

“福音”书虽有四部,但所传述的“福音”却只是一个,因为四圣史所撰述的是同一的喜讯,只是在所采用形式上有所不同而已。前三部《福音》,无论是在取材和结构上,或在用字上,大致可以说相同,甚至可并列对照,一望而知彼此间所有的关系,因而有“对观福音”之称。这三部《福音》之所以如此相同,是因为前三圣史记述了大体相同的“宗徒教理讲授”:\UL[玛窦]记述了\UL[耶路撒冷]教会的传授,\UL[马尔谷]记述了\UL[罗马]教会的传授,\UL[路加]记述了\UL[安提约基雅]教会的传授。\UL[若望]因见前三《福音》已流传于世,没有重述的必要,遂由自己记忆所及,采取了一些有关的材料,在第一世纪末叶,针对当时人事环境的需要,编著了自己的《福音》,其目的是在攻击方兴的异端邪说。

四《福音》虽然不是狭义的史书,但就信实性来说:世界上没有一部史书可与之相比,因为各位作者,或是目睹所述之事的宗徒(\UL[玛窦]、\UL[若望]),或是宗徒的亲传弟子(\UL[马尔谷]、\UL[路加]),他们所依据的,全是亲历其事人物的口述;况且《福音》成书时,尚有不少耳闻目睹的证人生存于世。

四《福音》内不但包含了有关信仰绝对重要的道理,而且也给世人提示了诸德的完美模范,基督徒成全的最高理想:即为我们降生成人的天主圣子。


\chapter*{玛窦福音引言}
\addcontentsline{toc}{chapter}{玛窦福音引言}
第一部《福音》的作者是圣\UL[玛窦]宗徒。\UL[玛窦]又名\UL[肋未],是\UL[阿耳斐]的儿子(\uwave{谷}2:14)。他在\UL{耶稣}召叫之前,曾在\UL[葛法翁]作过税吏。他一被召,即刻舍弃一切,跟随了\UL{耶稣}(\uwave{玛}9:9;\uwave{谷}2:13、14;\uwave{路}5:27、28)。\UL{耶稣}升天后,他先在\UL[巴力斯坦]一带,给自己的同胞宣讲福音多年,然后动身往外方传教去了。最后死在何处何时,史无确证。圣教会从古以来,即认他为一位为主殉道的宗徒,每年九月二十一日庆祝他的瞻礼。

据最古的传授,圣教会始终认为圣\UL[玛窦]是第一部《福音》的作者;这也可由《福音》书内的暗示得到证明:例如\UL[马尔谷]与\UL[路加]记载十二位宗徒名单时,只记了\UL[玛窦]的名字,然而在第一部《福音》内,于“\UL[玛窦]”名字前却加上了受人歧视的“税吏”头衔,可知原作者对自己的职位,毫不避讳。

《玛窦福音》的原著为\UL[阿刺美]文,因为是为\UL[巴力斯坦]的\UL[犹太]人写的,这是自古以来圣教会一致公认的事。此书后来不知由何人译为\UL[希腊]文。本《福音》因为是写给归化的\UL[犹太]人,因此特别为证明\UL[耶稣]\UL[基督]即是天主所预许及先知所预言的“默西亚”。虽然大多数\UL[犹太]人否认\UL[耶稣]为默西亚,并把他置于死地:然而他却由死者中光荣复活,并建立了自己的教会作为天国在世上的开端,继续他救世的使命。由于这个特殊的目的,\UL[玛窦]比其他三位圣史,更强调先知们的预言在\UL[耶稣]身上全应验了。

本书的著作地点,大概是\UL[耶路撒冷]。至于著作时期,原文可说是写于其他《福音》之前,大约著于公元50年左右;现行的\UL[希腊]译本,大概是成于《马尔谷》和《路加》两福音之后,约在公元70年左右。

本书记述\UL[耶稣]言行,并未全按编年的次第,而是出于作者的匠心独运。他把\UL[耶稣]公开传教的整个生活分作五段,每段先记事,后记言。此五段即是:(一)3-7;(二)8-10;(三)11-13:53;(四)13:54-18;(五)19-25。

本《福音》因是四《福音》中材料最丰富的一部,在结构上又是最有系统的一部,为此本《福音》在教会内应用最广,引用最多。


\chapter{玛窦福音}


\section{耶稣童年史(1,2)}


\subsection{第一章 族谱}