\part{旧约}


\chapter*{新旧约全书总论}
《新旧约全书》是数十卷经书的总集。这些经书的特点,在于它们的写成有超乎自然之处,因为这些经书都是在天主对神默感下写成。赐予天主的子民——教会——的礼品。

圣教会自古以来,一致主张这部总集包括《旧约》46卷,《新约》27卷,共计73卷。但大多数的\UL[基督教]派,由于只相信以\UL[希伯来]文写成的书才为圣经,因此现今只有\UL[希腊]原文的《巴路克》、《多俾亚传》、《友弟德传》、《玛加伯》上下、《智慧篇》和《德训篇》7卷,未著录在他们的圣经书目内。而天主教会自古即以\UL[希腊]文七十贤士本为圣经,因而对上述7卷也一律认为是圣经。

这些经书称为“约“,因为其中心思想,是天主与人类所立的盟约。天主与\UL[以色列]民族在\UL[西乃]山上所立的盟约,称为“旧约“;\UL[耶稣]以自己的圣血和圣死为全人类所立的永远盟约,称为“新约“。这些经书又称为“圣经“,是为表示这些书所具有的独特地位和神圣权威。书中所记述的一切,是吾人信仰及道德的大经,又为吾人立身经世的大道。

《旧约》经书的原文,除几卷和几小段外,大都以\UL[希伯来]文写成。后来侨居\UL[北非]受了\UL[希腊]文件影响的\UL[犹太]人,因多不谙\UL[希伯来]文,\UL[犹太]人遂在公元前三至二世纪,将《旧约》各书译为\UL[希腊]文,即今所称的“七十贤士译本“。以后\UL[希腊]文\footnote[0]{原书为“语文“}也成了\UL[罗马]帝国的通用语言,宗徒们在各宣讲福音,为了方便起见,即时常利用这部\UL[希腊]文圣经。为此这部\UL[希腊]文圣经(包括46卷)自初即为教会所尊重,并具有极大的权威。

《新约》各书,全部是以\UL[希腊]文写成,只有《玛窦福音》,原文虽为\UL[阿辣美]文,但很早即已失传;今所留传的,只有\UL[希腊]文本。

《旧约全书》的写成,凡经一千余年(约由公元前1300至100年),而逐渐汇为一集。《新约全书》是公元初世纪宗徒时代的作品。

《新旧约全书》,通常分为三大类:即历史书、先知书和智慧书(或训诲书),这是很广泛的分类。至于作者,《旧约》大多出于先知及其他贤哲的手笔,《新约》是宗徒和宗徒弟子的写作。但因全部圣经都是“因天主的默感写成的“(\uwave[弟]后3:16),经内的话是“由天主所派遣的圣人,在圣神推动之下说出来的“(\uwave[伯]后),为此我们不得不承认圣经的首要作者是天主。所谓“默感“,即是说:圣经的作者与编者(人),在天主的灵性感动之下,写下天主愿向他的子民(旧约与新约的教会)所要说的话,记下天主要他们记述的史事。有时天主也曾向他们透露某些重要的事迹,或直接向他们说话;这样,作者不仅获得了“默感“,同时也获得了“启示“。既然天主是圣经的首要作者,那么圣经上所记载的即是天主的话,即是天主的“圣言“。既然是天主的话,那么圣经上所载的一切,句句都是“真实无误“。就是说:圣经作者在天主默感下所愿表示出来的意义,是不会错误的。但为了解作者所要表达的本意,必须先注意经书中每部书的文体和体裁:是散文或是诗体?是历史或是传奇?是寓言或是训诲?因为每种文体有其独特的意义。同时还应注意作者或编者的时代背景,因为时代不同论事的观点也各有异。比如古代民族,尤其\UL[以色列]人对历史的观点,和今日的史学家的观点,有绝大的不同。尤其圣经的作者或编者,是本着宗教观点来编述历史的过程。他们看历史时,常着眼于天主为历史的推动者和支配者;人民的盛衰兴亡,常系之于他们是否遵守天主的法律。

另一个极重要的问题,是圣经与科学。圣经的作者决无意又教授自然科学(如宇宙学、天文学、生物学、人类学等)为写作的目的。圣经作者的目的,是在于启迪人类“获得拯救的智慧”(\uwave{弟}后3:15);为此他们无意研究自然界的进化和人体的构造,其用意识在说明自然界和人类与天主的关系,教导世人,天地万物都来自天主,一切都因天主的照顾而生存,最后又归于天主。

还就当注意的是:为适当地研究圣经和解释经意,人人必须先有信仰,并甘心接受圣教会的指导,因为天主把圣经委托给教会保管,因此只有教会才有解释圣经的特权。

圣经中所记载的都是些最重要的真理。教父多称圣经为天主给流徙的世人寄来的家书或天书。在这部天书内,天主先将自己启示给世人并告知世人,天地万物的来历和目的,告诉我们天主原先怎样给人类备下了幸福,现今的痛苦、患难和死亡又怎样来的;天主在漫长的历史过程中,怎样逐步实现了他救赎人类的大计划。《旧约》》所载,即是天主为\UL[以色列]民族所行的大事,所定的法律,所发的劝言和警告。这一切都是他为完成救赎人类工程的准备,甚至\UL[以]民的被选,也是为准备万民获得救恩。《新约》则是全人类得救的大喜讯,记载着人类唯一的救主\UL[耶稣]\UL[基督]救赎人类的大事;因为\UL[耶稣]是天主第二位圣子,只有他才能把天主性的真理启示给人类。他降生成人,藉着自己的人性,完成了天主慈爱的计划,使人与天主重归于好,且提高了人类的地位,使人分享天主性的永福。

由此看来,\UL[耶稣]\UL[基督]实为《旧新》二约的中心,是“法律”(旧约)的终向(\uwave[罗]10:4),是“新约的中保”(\uwave[希]8:6)。《旧新》二约各经卷的最后目的,就是叫人准备期待“我们伟大的天主及救主耶稣基督的光荣显现”(\uwave[铎]2:13)。

圣经为人类得救既有如此重要的关系,因此圣经对圣教会,对于一切基督信徒,对全人类,的确是举世无双的无价宝书。圣\UL[保禄]论《旧约》说:“凡所写的,都是为教训我们写的”(\uwave{罗}15:4),“为教训,为督责,为矫正,为教导人学正义,都是有益的”(\uwave{弟}后3:16)。这些话对《新约》来说,更为恰当;而且可说,如果没有圣经,我们无论对天主,或是对人,不会有一个正确的认识,因为只凭理性的自然神学是不够的,决不能打动人心,惟有研读圣经才能触及我们灵魂的深处,使我们听得见“生活天主”的话,领略天主威爱兼有的声音,洞见他全能的伟大化工,明白他怎样生养保存万物,怎样怎样以他至高无上的主权宰治一切,裁判一切;又怎样以他交慈父心肠,导引迷途的荡子,回归父家。无怪乎教宗\UL[良]十三称圣经为“神学的灵魂”。

诚然,一个怀有信德的教友,在恭读默思圣经时,应觉得是与天主会晤,是在静听天主的劝导,是在听他在天之父的慈音。当他心有所得,情有所动时,自然就向天主说话,这即是祈祷。无怪乎圣教会自古即以圣经为赞美、祈祷、默想最好的宝书。信友如能日日如此读经,与天主互诉衷曲,在日常的生活上或工作上,性能时时对越天主,承行他的圣意,臻于圣化一切的至境。为此教会不断劝勉信友多读圣经,尤其这次大公会议,对圣经研究与圣经诵读特予强调。愿我信友善体慈母教会的劝告,勉力天天去阅读这部天赐宝书。


\chapter*{梅瑟五书引论}
《旧约全书》前五卷,通称“梅瑟五书”,或简称“五书”。因为在这五卷书内,包含着《旧约》中最重要的一部分,即\UL[梅瑟]给\UL[以色列]人所宣布的法律,为此圣经上多次称《五书》为“法律”。\UL[希伯来]人称之为“托辣”。

《梅瑟五书》虽然在纪元前已有如此的分划:即《创世纪》、《出谷纪》、《肋未纪》、《户籍纪》、《申命纪》,但自古以来这五卷书常视为一部,且是一部世界文学上的杰作。

如果说全部圣经的主题是阐述人类的救赎史,那么“五书”即是记述这救赎史的开端。作者从天地和人类的创造开始,说到人类因违背天主的命令,而失掉原有的幸福,再扼要地叙述各民族的太古史,继而只着重于\UL[以色列]民族的起源,及其成为天主选民的历史。这历史的中心即在于天主与选民在\UL[西乃]山上所结的盟约。天主很早即对\UL[以色列]人的先祖再三地许诺,要以特殊照顾和非凡的奇事,准备\UL[以]已的心灵,使他们对天主养成坚定不移的信仰,以后好藉\UL[梅瑟]选立他们为自己的国民,颁下当遵行的法律,在世上建立起神权政体的神国。以后又在旷野四十年之久,以种种试探考验了他们的忠诚和信心;最后引他们到了\UL[约但]河东岸,在那里又藉\UL[梅瑟]劝告他们,重述以前教导过他们的一切,准备他们进占已预许给他们祖先的福地。所以从历史方面来看,“五书”有其统一的目标,实是一部上下一贯的著述。

按古来一致的传授,“五书”的作者是\UL[梅瑟]。称他为作者,并不是说全书每字每句都出于他一人的手笔,而是说他曾搜集了不少当时所能找得到的史料、文献和法律。且在他死后,有许多历史或法律部分是后人增补的,因为“五书”原是\UL[以色列]人宗教、政治、社会生活的法典,所以常有随时增添一些解释的必要,为使\UL[梅瑟]法律能随历史的演变,而适应时代的环境。

从以上所述,可知“五书”为\UL[以色列]人具有多么重要的关系。如果我们对“五书”没有认识,便不能明了\UL[以色列]子已的历史,因为他们生活在一个神权政体的制度之下,他们的存亡盛衰,全系于他们是否忠实履行天主藉\UL[梅瑟]所颁布的法律。在《旧约》其他经书内,常不断指出法律的这种重要关系,并依法律为原则,来批判一切历史的得失。但这法律的最终目的,诚如圣\UL[保禄]所说:“法律的终向本来是基督”(\uwave{罗}10:4)。为此,法律为\UL[以色列]人,好像是“归于基督的启蒙师”(\uwave{迦}3:24)。换句话说,法律应领导\UL[以色列]人,认识并信仰将要来临的默西亚。当默西亚\UL[耶稣]\UL[基督]一降生,法律的使命就算完结,\UL[耶稣]所宣讲的“爱的诫命”,满全了整个法律(\uwave{罗}13:10)。虽然如此,“五书”为《新约》的教会,仍未失其重要性,因为本书含有永生天主的启示,以及教会信仰的基础。


\chapter*{创世纪引言}
“梅瑟五书”每部的名称,\UL[犹太]人皆以每书的首句首字为名。自\UL[希腊]七十贤士译经以来,皆以每书的内容大意命名。“梅瑟五书”的第一部名为《创世纪》,因为本书并不是以科学的论点和近代史学家的方法来记述,而是本着宗教的观点来说明救赎史的开端。在这救赎史中,依照天主的计划,\UL[以色列]民族在万民中占着重要的角色,因此作者也只着重于这个民族的历史。

本书的前编(1-11章),是救世史的前导,说明天主是整个天地万物的创造者,和全人类历史的领导者;并指出\UL[以色列]人与其他民族的关系。原父母虽然背命,惹下了滔天大祸,后继的人们也多半背弃了天主,如洪水和\UL[巴贝耳]塔时代的人,但因为人是按天主的肖像造成的,天主决不愿将全人类完全抛弃,所以在后编内(12-50章),作者便记述天主怎样拣选了一位信仰坚定,服从听命的人——\UL[亚巴郎],怎样向他起誓,立他为一个新民族的始祖,即将来要成为天主选民的民族的始祖,许下因他和他的后裔,天下万民将要获得祝福(22:18),由他的后裔中要生出一位“应得权杖,万民都要归顺他”(49:10),他要使“救恩达于地极”的后裔(\uwave{依}49:6)。在记述\UL[亚巴郎]、\UL[依撒格]、\UL[雅各伯]和\UL[若瑟]的事迹中,作者一再证明天主怎样特殊地照顾了他们,以准备救赎人类的道路。


\chapter{创世纪}


\section{前编  太古史(1-11)}


\subsection{第一章 天地万物的创造}
\renewcommand{\thefootnote}{\alph{footnote}}

$^{1}$在起初天主创建了天地。$^{2}$大地还是混沌空虚,深渊上还是团黑暗,天主的神在水面上运行。\footnote{1 “在起初……”一语,暗示创造万物之时,除天主外,一无所有。“天地”二字此处有宇宙万物之意。作者用诗人的相像力描写天主好似一个工程师,在六天以内创造了万物,到第七天休息。首先所创造的是混沌的无生之物,后将这混沌之物分成天、地、海三大部分,然后以日月、星辰、草木、飞禽、走兽等来点缀天地海洋。最后天主照自己的肖像造了人。作者从创造混沌之物说起,到创造人,表示人是万物之灵,应效法造物主工作和守安息日。此开宗明义第一章是远古时代的文学杰作,是一篇宗教的重要文告,并不是自然科学的论著。按古代各民族对天地开辟,人类诞生的传说,没有可与《创世纪》第一章相比拟的。“天主的神”指施生命之神力,但若通观新旧二约的全部启示,此处也指赐生命的“天主圣神”。}$^{3}$天主说:“有光!”就有了光。$^{4}$天主见光好,就将光与黑暗分开。$^{5}$天主称光为“昼”,称黑暗为“夜”。过了晚上,过了早晨,这是第一天。

$^{6}$天主说:“在水与水之间要有穹苍,将水分开!”事就这样成了。$^{7}$天主造了穹苍,分开了穹苍以下的水和穹苍以上的水。$^{8}$天主称穹苍为“天”,天主看了认为好。过了晚上,过了早晨,这是第二天。

$^{9}$天主说:“天下的水应聚在一处,使旱地出现!”事就这样成了。$^{10}$天主称旱地为“陆地”,称水汇合处为“海洋”。天主看了认为好。$^{11}$天主说:“在陆地上,土地要生出青草、结种子的蔬菜和结果子的果树,各按照在它内的种子的种类!”事就这样成了。$^{12}$土地就生出了青草,结种子的蔬菜,各按其类,和结果子的树木,各按照在它内的种子的种类。天主看了认为好。$^{13}$过了晚上,过了早晨,这是第三天。

$^{14}$天主说:“在天空中要有光体,以分别昼夜,作为规定时节和年月日的记号。$^{15}$要在天空中放光,照耀大地!”事就这样成了。$^{16}$天主于是造了两个大光体:较大的控制白天,较小的控制黑夜,并造了星宿。$^{17}$天主将星宿摆列在天空,照耀大地,$^{18}$控制昼夜,分别明与暗。天主看了认为好。$^{19}$过了晚上,过了早晨,这是第四天。

$^{20}$天主说:“水中要繁生蠕动的生物,地面上、天空中要有鸟飞翔!”事就这样成了。$^{21}$天主于是造了大鱼和所有在水中孳生的蠕动生物,各按其类,以及各种飞鸟,各按其类。天主看了认为好。$^{22}$遂祝福它们说:“你们要孳生繁殖,充满海洋;飞鸟也要在地上繁殖!”$^{23}$过了晚上,过了早晨,这是第五天。

$^{24}$天主说:“地上要生出生物,各按其类;走兽、爬虫和地上的各种生物,各按其类!”事就这样成了。$^{25}$天主于是造了地上的生物,各按其类;各种走兽,各按其类;以及地上所有的爬虫,各按其类。天主看了认为好。$^{26}$天主说:“让我们照我们的肖像,按我们的模样造人,叫他管理海中的鱼、天空的飞鸟、牲畜、各种野兽、在地上爬行的各种爬虫。”$^{27}$天主于是照自己的肖像造了人,就是照天主的肖像造了人:造了一男一女。$^{28}$天主祝福他们说:“你们你们要生育繁殖,充满大地,治理大地,管理海中的鱼、天空的飞鸟、各种在地上爬行的生物!”\footnote{1 “人”按原文有红土或黄土的意思,是说人是属于土的造物。“我们”(26节)按古\UL[犹太]经师的解释,是指天主和天使,好似天主同天使商量;但有些学者主张为“威严复数”或“议决复数”。教父和神学家多以为此复数暗示天主圣三的奥理。此说若照启示的演进说是对的。人相似天主是按灵魂说的,相似天主有理智、意志和记忆。论人的肉身,当天主造\UL[亚当]时,已预见作\UL[亚当]第二的\UL[基督](\uwave{罗}5:14)。“造了一男一女”,指婚姻一夫一妻制和不可分离性(\uwave{玛}19:1-6;\uwave{拉}2:15、16)。天主祝福原祖生育繁殖的话,说明婚姻的首要目的是生养教育子女(8:17;\uwave{咏}127:3、4)。}$^{29}$天主又说:“看,全地面上结种子的各种蔬菜,在果内含有种子的各种果树,我都给你们作食物;$^{30}$至于地上的各种野兽,天空中的各种飞鸟,在地上爬行有生魂的各种动物,我把一切青草给它们作食物。”事就这样成了。$^{31}$天主看了他造的一切,认为样样都很好。过了晚上,过了早晨,这是第六天。\footnote{1 天主造了原祖,也赐给了他们和他们传生的人类食物,并将普世交给他们统治。所造的万物样样都好,是说万物都合天主的旨意,都为他所喜爱。参阅\uwave{咏}19:1-6,104,145,148,150。}


\subsection{第二章 安息日}
$^{1}$这样,天地和天地间的一切点缀都完成了。$^{2}$到第七天天主造物的工程已完成,就在第七天休息,停止了所作的一切工程。$^{3}$天主祝福了第七天,定为圣日,因为这一天,天主停止了他所行的一切创造工作。\footnote[1]{2 1-3节属前章,劝人守安息日为圣日。守安息日的原因与目的,见\uwave{出}23:12;\uwave{申}5:12-15。}


\subsubsection{人与乐园}
$^{4}$这是创造天地的来历:在上主天主创造天地时,$^{5}$地上还没有灌木,田间也没有生出蔬菜,因为上主天主还没有使雨降在地上,也没有人耕种土地,$^{6}$有从地下涌出的水浸润所有地面。$^{7}$上主天主用地上的灰土形成了人,在他鼻孔内吹了一口生气,人就成了一个有灵的生物。$^{8}$上主天主在\UL[伊甸]东部种植了一个乐园,就将他形成的人安置在里面。$^{9}$上主天主使地面生出各种好看好吃的果树,生命树和知善恶树在乐园中央。\footnote{2 2:4-3:24为创造天地万物的另一记载。原来在这记载中只用了上主(雅威)的名词,但将这个记载与上章的记载编在一起时,补入了“天主”的名词。这记载的中心为人:天主对人,人对天主的态度。关于人的来历和本性,作者用简略的话,教训人一端论宗教和文化的最高深的道理:人肉身的形成,好像其他的动物,是由尘土造成的,但对于灵魂却有极大的区别,它是直接由天主所造。\UL[伊甸]乐园位于何处,人不得而知。乐园是天主考验人的地方。“生命树”所象征的是天主愿意赐给人的“不死”之恩。“知善恶的树”,是试探人的工具。“知善恶”的意思,大概是说:人一犯天主的禁令,就知道所失去的超性恩宠——真善,是多么完善,所犯的罪恶——真恶,是如何凶恶。}$^{10}$有一条河由\UL[伊甸]流出灌溉乐园,由那里分为四支:$^{11}$第一支名叫\UL[丕雄],环流产金的\UL[哈威拉]全境;$^{12}$那地方的金子很好,那里还产珍珠和玛瑙;$^{13}$第二支河名叫\UL[基红],环流\UL[雇士]全境;$^{14}$第三支河名叫\UL[底格里斯],流入\UL[亚述]东部;第四支河即\UL[幼发拉的]。$^{15}$上主天主将人安置在\UL[伊甸]的乐园内,叫他耕种,看守乐园。\footnote{2 说明人犯罪之前,天主已叫人应该工作。}$^{16}$上主天主给人下令说:“乐园中各树上的果子,你都可吃,$^{17}$只有知善恶树上的果子你不可吃,因为那一天你吃了,必定要死。”


\subsubsection{造女人立婚姻}
$^{18}$上主天主说:“人单独不好,我要给他造个与他相称的助手。“$^{19}$上主天主用尘土造了各种野兽和天空中的各种飞鸟,都引到人面前,看他怎样起名;凡人给生物起的名字,就成了那生物的名字。$^{20}$人遂给各种畜牲、天空中的各种飞鸟和各种野兽起了名字;但他没有找着一个与自己相称的助手。$^{21}$上主天主遂使人熟睡,当他睡着了,就取出了他的一根肋骨,再用肉补满原处。$^{22}$然后上主天主用那由人取来的肋骨,形成了一个女人,引她到人前,$^{23}$人遂说:“这才真是我的骨中之骨,肉中之肉,她应称为“女人“,因为是由男人取出的。“$^{24}$为此人应离开自己的父母,依附自己的妻子,二人成为一体。$^{25}$当时,男女二人都赤身露体,并不害羞。\footnote{2 本段的要义有二:一、人给动物命名,是表示人受有统治一切造物的权柄;二、从\UL[亚当]的肉身形成了第一个女人,是指女人同他有一样的人性,像\UL[亚当]一样是照天主的肖像受造的。夫妇结为一体,表示婚姻的结合是天主制定的,人不能拆散(\uwave{玛}19:5、6)。赤身不害羞,是说原祖未犯罪前纯洁无罪的状态,还未体验到罪过的恶果。}


\subsection{原祖违命}
$^{1}$在上主天主所造的一切野兽中,蛇是最狡猾的。蛇对女人说:“天主真说了,你们不可吃乐园中任何树上的果子吗?“$^{2}$女人对蛇说:“乐园中树上的果子,我们都可吃;$^{3}$只有乐园中央那棵树上的果子,天主说过,你们不可以吃,也不可摸,免得死亡。“$^{4}$蛇对女人说:“你们决不会死!$^{5}$因为天主知道,你们那天吃了这果子,你们的眼就会开了,将如同天主一样知道善恶。“$^{6}$女人看棵果树实在好吃好看,令人羡慕,且能增加智慧,遂摘下一个果子吃了,又给了她的男人一个,他也吃了。$^{7}$于是二人的眼立即开了,发觉自己赤身露体,遂用无花果树叶,编了个裙子围身。$^{8}$当\UL[亚当]和他的妻子听见了上主天主趁晚凉在乐园中散步的声音,就躲藏在乐园的树林中,怕见上主天主的面。\footnote{3 本章记的蛇就是魔鬼。他藉蛇形诱惑了\UL[厄娃](\uwave{智}2:23、24;\uwave{若}8:44;\uwave{默}12:9,20:2)。原祖所犯是骄傲背命的罪。“发觉自己赤身“,是指失去天主的宠爱和原始的纯洁。}$^{9}$上主天主呼唤\UL[亚当]对他说:“你在哪里?“$^{10}$他答说:“我在乐园中听到了你的声音,就害怕起来,因为我赤身露体,遂躲藏了。“$^{11}$天主说:“谁告诉了你,赤身露体?莫非你吃了我禁止你吃的果子?“$^{12}$\UL[亚当]说:“是你给我作伴的那个女人给了我那树上的果子,我才吃了。“$^{13}$上主天主遂对女人说:“你为什么作了这事?“女人答说:“是蛇哄骗了我,我才吃了。“\footnote{3 天主询问时,没有询问魔鬼,只询问了\UL[亚当]\UL[厄娃];但惩罚时却按罪过的原因和轻重;先是魔鬼,后是\UL[厄娃],最后是\UL[亚当](\uwave{弟}前2:13-15)。}


\subsubsection{处罚与预许}
$^{14}$上主天主对蛇说:“因你做了这事,你在一切畜牲和野兽中,是可咒骂的;你要用肚子爬行,毕生日日吃土。$^{15}$我要把仇恨放在你和女人,你的后裔和她的后裔之间,她的后裔要踏碎你的头颅,你要伤害他的脚跟。”$^{16}$后对女人说:“我要增加你怀孕的苦楚,在痛苦中生子;你要依恋你的丈夫,也要受他的管辖。”$^{17}$后对\UL[亚当]说:“因为你听了你妻子的话,吃了我禁止你吃的果子,为了你的缘故,地成了可咒骂的;你一生日日劳苦才能得到吃食。$^{18}$地要给你生出荆棘和蒺藜,你要吃田间的蔬菜;$^{19}$你必须汗流满面,才有饭吃,直到你归于土中,因为你是由土来的;你既是灰土,你还要归于灰土。”\footnote{3 天主的传递既超过了他的公义,故此他在义怒中给人类预许了人终要得胜魔鬼的诺言;因这许诺,3:15称为“原始福音”。大义是:踏碎蛇头是得胜魔鬼的象征;“女人的后裔”虽然也指犯罪败坏的人类,但在此特别指拯救人类的新元首\UL[基督](\uwave{哥}1:15-18),只有他打败了魔鬼;故此圣\UL[保禄]称他为“新\UL[亚当]”(\uwave{罗}5:12-15)。魔鬼同\UL[厄娃]的对白,与天使同\UL[玛利亚]的对白恰恰相反:一是诱惑的对白,一是商讨救赎的对白;因此教父由第2世纪起即称\UL[玛利亚]为“新\UL[厄娃]”。又因她与\UL[基督]的密切结合,她也踏碎了魔鬼的头颅。圣母始胎无玷的道理,由此处已露曙光(\uwave{路}1:26-38;\uwave{默}12)。}


\subsubsection{被逐出乐园}
$^{20}$\UL[亚当]给自己的妻子起名叫\UL[厄娃],因为她是众生的母亲。$^{21}$上主天主为\UL[亚当]和他的妻子做了件皮衣,给他们穿上;$^{22}$然后上主天主说:“看,人已相似我们中的一个,知道了善恶;如今不要让他伸手再摘取生命树上的果子,吃了活到永远。”上主天主遂把他赶出\UL[伊甸]乐园,叫他耕种他所由出的土地。$^{24}$天主将\UL[亚当]逐出了以后,就在\UL[伊甸]乐园的东面,派了“革鲁宾”和刀光四射的火剑,防守到生命树去的路。\footnote{3 “革鲁宾”按\UL[巴比伦]语有“保护者”之意(\uwave{出}25:18-22;\uwave{则}1:11)。}


